\tolerance=10000
\documentclass{article}
\usepackage{amsmath,times,graphicx,url}

\newcommand{\pd}[2]{\frac{\partial #1}{\partial #2}}
\newcommand{\bq}{{\bf q}}
\newcommand{\bx}{{\bf x}}
\newcommand{\bu}{{\bf u}}
\newcommand{\bw}{{\bf w}}
\newcommand{\bz}{{\bf z}}
\newcommand{\bK}{{\bf K}}
\newcommand{\balpha}{{\bf \alpha}}
\newcommand{\bbeta}{{\bf \beta}}
\newcommand{\avg}[1]{E\left[ #1 \right]}
\DeclareMathOperator*{\argmin}{argmin}
\DeclareMathOperator*{\sgn}{sgn}

\usepackage{xcolor}
% switch to other comment command to remove all comments.
\newcommand{\mycomment}[2]{\textcolor{red}{#1}\footnote{\textcolor{red}{#2}}}
%\newcommand{\mycomment}[1]{#1}

\begin{document}

First, define $q = [q_p^T,q_a^T]^T$, where $q_p$ are the passive
(floating base) joints, and $q_a$ are the actuated joints.  If I
assume that the ground contacts on the feet will stay stationary, then
I have \begin{gather*} \dot{x}_{gc} = [ J_{gc}^p, J_{gc}^a
  ] \begin{bmatrix} \dot{q}_p \\ \dot{q}_a \end{bmatrix} = 0, \\
  \Rightarrow \quad J_{gc}^p \dot{q}_p = - J_{gc}^a \dot{q}_a, \\
  \Rightarrow \quad \dot{q}_p = -J_{gc}^{p+} J_{gc}^a \dot{q}_a, \\
  \Rightarrow \quad \dot{q} = \begin{bmatrix} -J_{gc}^{p+} J_{gc}^a,
    \\ I \end{bmatrix} \dot{q}_a = A \dot{q}_a, \end{gather*} where
$J_{gc}$ is the kinematic Jacobian of the ground contacts, the
superscripts $a$ and $p$ denote the subsets of this matrix related to
$q_a$ and $q_p$, and the superscript $+$ indicates a pseudo-inverse.
In the planar case, $J_{gc}^p$ will likely have full column rank if
there are two contact points (even on a single foot); in the 3D case
you'd like to have three contact points (presumably on at least two
different bodies).

For control, I want to regulate the $x$ and $y$ positions of the center of mass (first priority), then regulate hand position (second priority), then stay near a comfortable posture (last priority).  For a good summary of kinematic Jacobian control, see \cite{Siciliano90}.  We will use 
\begin{gather*} 
\dot{x}_{com,xy} = \begin{bmatrix} I & 0 \end{bmatrix} J_{com} A \dot{q}_a = J_1 \dot{q}_a, \quad x_{com,xy}^d = x_{csp}, \text{ and} \\
\dot{x}_{rhand} = J_{rhand} A \dot{q}_a = J_2 \dot{q}_a,
\end{gather*} 
where COM stands for ``center of mass'' and CSP stands for ``center of support polygon''.  Set \begin{gather*} \dot{x}_1 = \eta_1 (x_{com,xy}^d - x_{com,xy}) \\ 
\dot{x}_2 = \eta_2 (x_{rhand}^d - x_{hand}) \\ \dot{q}_0 = \eta_3 (q_{0}^d - q_0) \end{gather*} Finally, the control law is given by (\cite{Siciliano90} eq 15):
\[ q_d = q + J_1^+ \dot{x}_1 +  (I - J_1^+ J_1) J_2^+ (\dot{x}_2 - J_2 J^1+ \dot{x} ) + (I - J_1^+ J_1) (I - J_2^+ J_2) \dot{q}_0. \]




\bibliographystyle{plain}
\bibliography{elib}

\end{document}
